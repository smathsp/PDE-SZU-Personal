\section{二阶 PDE 的分类}

\subsection{特征线与特征方程}
\[
a u_{xx} + 2b u_{xy} + c u_{yy} + d u_x + e u_y + g u = f
\]

\[
a \, u_{xx} + 2b \, u_{xy} + c \, u_{yy} + \cdots = 0 \rightarrow a \left( \frac{dy}{dx} \right)^2 - 2b \frac{dy}{dx} + c = 0
\]

定义 $\Delta = b^2 - ac$,根据 $\Delta$ 的取值,可以将方程分类为:
\begin{itemize}
    \item $\Delta > 0$:双曲型,存在两条实特征线,为波动方程
    \item $\Delta = 0$:抛物型,存在一条实特征线,为热传导方程
    \item $\Delta < 0$:椭圆型,不存在实特征线,为调和方程
\end{itemize}


\subsection{解法}

首先找出$a,b,c$解出$\Delta = b^2 - ac$根据$\Delta > 0$为双曲形,$\Delta = 0$为抛物形,$\Delta < 0$为椭圆形分类讨论

得到$\xi \eta$后计算偏导数替换原式所有$xy$即可

\subsubsection{双曲形}

\[
\pm \lambda = \frac{b \pm \sqrt{\Delta}}{a}
\]

\[
\left\{
\begin{aligned}
\xi &= y - \lambda_1 x \\
\eta &= y - \lambda_2 x
\end{aligned}
\right.
\]

\subsubsection{抛物形}

\[
\left\{
\begin{aligned}
\xi &= y - \frac{b}{a} x \\
\eta &= y
\end{aligned}
\right.
\]

\subsubsection{椭圆形}

\[
\left\{
\begin{aligned}
\bar{\xi} &= y - \frac{b}{a} x \\
\bar{\eta} &= -\frac{ac-b^2}{a} x
\end{aligned}
\right.
\]

\subsection{例题}
判断以下偏微分方程的类型,并将其化为标准型

\subsubsection*{题目}
\[
u_{xx} - y u_{yy} = 0
\]

\subsubsection*{解答}

\[
a = 1, \quad b = 0, \quad c = -y
\]

\[
\Delta = b^2 - ac = y
\]

分类讨论:

\textbf{当 $y > 0$(双曲形)}

\[
\frac{dy}{dx} = \pm \sqrt{y}
\]

\[
\int \frac{dy}{\sqrt{y}} = \pm \int dx \quad \Longrightarrow \quad 2\sqrt{y} = \pm x + c
\]

\[
\left\{
\begin{aligned}
2\sqrt{y} &= x + c_1, \\
2\sqrt{y} &= -x + c_2
\end{aligned}
\right.
\quad \Longrightarrow \quad
\left\{
\begin{aligned}
\xi &= x + 2\sqrt{y}, \\
\eta &= x - 2\sqrt{y}
\end{aligned}
\right.
\]

\[
u_x = u_{\xi} + u_{\eta}, \quad u_y = \frac{1}{\sqrt{y}} ( u_{\xi} - u_{\eta} )
\]

\[
\left\{
\begin{aligned}
u_{xx} &= u_{\xi \xi} + u_{\eta \eta}, \\
u_{yy} &= -\frac{1}{2} y^{- \frac{3}{2}} u_{\xi} + \frac{1}{2} y^{-\frac{3}{2}} u_{\eta} + y^{-1} (u_{\xi \xi} + u_{\eta \eta})
\end{aligned}
\right.
\]

带入原式化简得到

\[
\frac{1}{2} y^{-\frac{1}{2}} (u_{\xi} - u_{\eta}) = 0
\]

由于\(\xi - \eta = 4y^{-\frac{1}{2}}\),带入得到:

\[
\frac{2}{\xi - \eta} (u_{\xi} - u_{\eta}) = 0
\]

\textbf{当 $y = 0$(抛物形)}