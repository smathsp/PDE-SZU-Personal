\section{分离变量求解}

\subsection{求解弦振动方程}

\[
\left\{
\begin{aligned}
u_{tt} - a^2 u_{xx} = 0, \quad 0<x<l, \quad t > 0, \\
u|_{t=0} = \varphi(x), \quad u_t|_{t=0} = \psi(x), \quad 0\leq x \leq l, \\
u|_{x=0} = 0, \quad u|_{x=l} = 0, \quad t \geq 0.
\end{aligned}
\right.
\]

第三行边值条件可能不同


\subsubsection*{题目}
求解如下弦振动方程的混合问题:

\setcounter{equation}{0}

\begin{align}
    u_{tt} - u_{xx} &= 0, \quad 0 < x < 1, \quad t \geq 0 \label{eq:1} \\
    u|_{t=0} &= \sin\left(\frac{3}{2} \pi x\right), \quad 
    u_t|_{t=0} = \sin\left(\frac{5}{2} \pi x\right), \quad 0 \leq x \leq 1 \label{eq:2} \\
    u_x|_{x=0} &= 0, \quad u_x|_{x=1} = 0, \quad t \geq 0 \label{eq:3}
\end{align}

\subsubsection*{解}

为了解此问题,我们采用分离变量法。将方程 \eqref{eq:1} 满足边界条件 \eqref{eq:3} 的解表示为:

\begin{equation}
    u(x, t) = X(x) T(t). \label{eq:4}
\end{equation}

将 \eqref{eq:4} 代入方程 \eqref{eq:3},得到:
\begin{equation}
    \frac{X''(x)}{X(x)} = \frac{T''(t)}{T(t)} = -\lambda, \nonumber
\end{equation}
其中 $\lambda$ 是一个常数。于是得到:
\begin{align}
    X''(x) + \lambda X(x) &= 0, \label{eq:5} \\
    T''(t) + \lambda T(t) &= 0. \label{eq:6}
\end{align}

由边界条件 \eqref{eq:3} 可知:
\begin{equation}
    X(0) T(t) = 0, \quad X'(1) T(t) = 0. \nonumber
\end{equation}

\begin{equation}
X(0) = X'(1) = 0 \label{eq:7}
\end{equation}

对于 \eqref{eq:5} 和 \eqref{eq:7},我们进行讨论:

1. 当 $\lambda < 0$ 时,方程 \eqref{eq:5} 的通解为:
   \[
   X(x) = c_1 e^{-\sqrt{-\lambda} \, x} + c_2 e^{\sqrt{-\lambda} \, x}.
   \]
   代入边界条件 $c_1 = c_2 = 0$,说明此时只有平凡解。

2. 当 $\lambda = 0$ 时,方程 \eqref{eq:5} 的通解为:
   \[
   X(x) = c_1 x + c_2.
   \]
   同样代入边界条件,得 $c_1 = c_2 = 0$,此时也只有平凡解。

3. 当 $\lambda > 0$ 时,方程 \eqref{eq:5} 的通解为:
   \[
   X(x) = c_1 \cos(\sqrt{\lambda} \, x) + c_2 \sin(\sqrt{\lambda} \, x).
   \]
   代入边值条件可得 $c_1 = 0 \quad c_2 \cos(\sqrt{\lambda}) = 0.$

   于是特征值为:
   \[
   \lambda_k = \left(k + \frac{1}{2}\right)^2 \pi^2, \quad k = 0, 1, 2, \dots
   \]
   对应的特征函数为:
   \[
   X_k(x) = c_k \sin\left(\left(k + \frac{1}{2}\right) \pi x\right), \quad k = 0, 1, 2, \dots
   \]

对于特征值 $\lambda_k$,解方程 \eqref{eq:6} 得:
\[
T_k(t) = a_k \cos\left(\left(k + \frac{1}{2}\right) \pi t\right) + b_k \sin\left(\left(k + \frac{1}{2}\right) \pi t\right),
\]
其中 $a_k$ 和 $b_k$ 是任意常数。于是对于任意$A_k=c_ka_k,B_k=c_kb_k$
\[
u_k(x, t) = \left(A_k \cos\left(\left(k + \frac{1}{2}\right) \pi t\right) + B_k \sin\left(\left(k + \frac{1}{2}\right) \pi t\right)\right) \sin\left(\left(k + \frac{1}{2}\right) \pi x\right).
\]
作级数
\[
u(x, t) = \sum_{k=1}^{\infty} \left(A_k \cos\left(\left(k + \frac{1}{2}\right) \pi t\right) + B_k \sin\left(\left(k + \frac{1}{2}\right) \pi t\right)\right) \sin\left(\left(k + \frac{1}{2}\right) \pi x\right).
\]
代入初值条件\eqref{eq:2},得到:
\[
\sum_{k=0}^{\infty} A_k \sin\left(\left(k + \frac{1}{2}\right) \pi x\right) = \sin\left(\frac{3}{2} \pi x\right),
\]
\[
\sum_{k=0}^{\infty} B_k \left(k + \frac{1}{2}\right) \pi \sin\left(\left(k + \frac{1}{2}\right) \pi x\right) = \sin\left(\frac{5}{2} \pi x\right).
\]

由此推得:
\[
A_0 = 0, \quad A_1 = 1, \quad A_2 = A_3 = \cdots = 0,
\]
\[
B_0 = B_1 = 0, \quad B_2 = \frac{2}{5\pi}, \quad B_3 = B_4 = \cdots = 0.
\]

所以混合问题\eqref{eq:1}-\eqref{eq:3}的解为:
\[
u(x, t) = \cos\left(\frac{3}{2} \pi t\right) \sin\left(\frac{3}{2} \pi x\right) + \frac{2}{5\pi} \sin\left(\frac{5}{2} \pi t\right) \sin\left(\frac{5}{2} \pi x\right).
\]