\section{偏微分方程的类型}

\subsection{线性偏微分方程(Linear PDEs)}
\textbf{定义}:未知函数 \(u\) 及其各阶导数在方程中线性出现,且没有乘积或非线性项。  

\begin{itemize}
    \item \textbf{未知函数及其导数的指数为 1}:
    \[
    u, \quad \frac{\partial u}{\partial x}, \quad \frac{\partial^2 u}{\partial x \partial y}
    \]
    这些项都只能以一次幂出现。

    \item \textbf{不能有未知函数的乘积}:  
    例如,像 \(u \cdot \frac{\partial u}{\partial x}\) 这样的项是非线性的。

    \item \textbf{系数可以依赖于自变量}(如 \(x\) 或 \(y\)),但不能依赖于未知函数 \(u\)。  
    例如:
    \[
    \frac{\partial u}{\partial x} + x u = 0
    \]
    是线性方程,但:
    \[
    \frac{\partial u}{\partial x} + u^2 = 0
    \]
    是非线性的。
\end{itemize}

\subsection{半线性偏微分方程(Semilinear PDEs)}
\textbf{定义}:最高阶导数线性出现,但非线性项涉及未知函数或其较低阶导数。

\textbf{示例}:
\begin{equation}
\frac{\partial u}{\partial t} - \Delta u = u^3,
\end{equation}
非线性项 \(u^3\) 只涉及未知函数。

\subsection{拟线性偏微分方程(Quasilinear PDEs)}
\textbf{定义}:最高阶导数线性出现,但系数可能依赖于未知函数或其较低阶导数。

\textbf{示例}:
\begin{equation}
u_t + u u_x = 0,
\end{equation}
这是无粘性伯格斯方程,最高阶导数项的系数依赖于 \(u\) \\
拟线性只能是 \(u u_{...}\) (u*偏导数)

\subsection{完全非线性偏微分方程(Fully Nonlinear PDEs)}
\textbf{定义}:最高阶导数本身非线性出现,无法写成线性或拟线性形式。